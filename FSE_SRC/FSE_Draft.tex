%
% The first command in your LaTeX source must be the \documentclass command.
\documentclass[sigconf,authordraft]{acmart}

%
% \BibTeX command to typeset BibTeX logo in the docs
\AtBeginDocument{%
  \providecommand\BibTeX{{%
    \normalfont B\kern-0.5em{\scshape i\kern-0.25em b}\kern-0.8em\TeX}}}

% Rights management information. 
% This information is sent to you when you complete the rights form.
% These commands have SAMPLE values in them; it is your responsibility as an author to replace
% the commands and values with those provided to you when you complete the rights form.
%
% These commands are for a PROCEEDINGS abstract or paper.
\copyrightyear{2019}
\acmYear{2019}
\setcopyright{acmlicensed}
\acmConference[ESEC/FSE '19]{27th ACM Joint European Software Engineering Conference and Symposium on the Foundations of Software Engineering}{August 26--30, 2019}{Tallinn, Estonia}
\acmBooktitle{27th ACM Joint European Software Engineering Conference and Symposium on the Foundations of Software Engineering , August 26--30, 2019, Tallinn, Estonia}
\acmPrice{15.00}
\acmDOI{10.1145/1122445.1122456}
\acmISBN{978-1-4503-9999-9/18/06}

%
% These commands are for a JOURNAL article.
%\setcopyright{acmcopyright}
%\acmJournal{TOG}
%\acmYear{2018}\acmVolume{37}\acmNumber{4}\acmArticle{111}\acmMonth{8}
%\acmDOI{10.1145/1122445.1122456}

%
% Submission ID. 
% Use this when submitting an article to a sponsored event. You'll receive a unique submission ID from the organizers
% of the event, and this ID should be used as the parameter to this command.
%\acmSubmissionID{123-A56-BU3}

%
% The majority of ACM publications use numbered citations and references. If you are preparing content for an event
% sponsored by ACM SIGGRAPH, you must use the "author year" style of citations and references. Uncommenting
% the next command will enable that style.
%\citestyle{acmauthoryear}

%
% end of the preamble, start of the body of the document source.
\begin{document}

%
% The "title" command has an optional parameter, allowing the author to define a "short title" to be used in page headers.
\title{FeRec: Automatic Features Recommendation Model  to support Bug Reporting}

%
% The "author" command and its associated commands are used to define the authors and their affiliations.
% Of note is the shared affiliation of the first two authors, and the "authornote" and "authornotemark" commands
% used to denote shared contribution to the research.
\author{Md. Rejaul Karim}
%\authornote{Both authors contributed equally to this research.}
\affiliation{%
  \institution{Nara Institute of Science and Technology (NAIST)}
  \city{Nara}
  \state{Japan}\\
\email{rejaul.karim.qw4@is.naist.jp}
\email{rejaul.karim.qw4@is.naist.jp}
 }



%
% By default, the full list of authors will be used in the page headers. Often, this list is too long, and will overlap
% other information printed in the page headers. This command allows the author to define a more concise list
% of authors' names for this purpose.
\renewcommand{\shortauthors}{Md Rejaul Karim, et al.}

%
% The abstract is a short summary of the work to be presented in the article.
\begin{abstract}
Bug reports are the primary means through which developers triage and fix bugs. To achieve this effectively, bug reports need to clearly describe those features that are important for the developers. However, previous studies have found that reporters do not always provide such features. Therefore, we first perform an exploratory study to identify the key features that reporters frequently miss in their initial bug report submissions. Then, we plan to propose an automatic approach for supporting reporters to make a good bug report. For our initial studies, we manually examine bug reports of five large-scale projects from two ecosystems such as Apache (Camel, Derby, and Wicket) and Mozilla (Firefox and Thunderbird). As initial results, we identify five key features that reporters often miss in their initial bug reports and developers require them for fixing bugs. We build and evaluate classification models using four different text-classification techniques. The evaluation results show that our models can effectively predict the key features. Our ongoing research focuses on developing an automatic features recommendation model to improve the contents of bug reports. 
\end{abstract}

%
% The code below is generated by the tool at http://dl.acm.org/ccs.cfm.
% Please copy and paste the code instead of the example below.
%
\begin{CCSXML}
<ccs2012>
 <concept>
  <concept_id>10010520.10010553.10010562</concept_id>
  <concept_desc>Computer systems organization~Embedded systems</concept_desc>
  <concept_significance>500</concept_significance>
 </concept>
 <concept>
  <concept_id>10010520.10010575.10010755</concept_id>
  <concept_desc>Computer systems organization~Redundancy</concept_desc>
  <concept_significance>300</concept_significance>
 </concept>
 <concept>
  <concept_id>10010520.10010553.10010554</concept_id>
  <concept_desc>Computer systems organization~Robotics</concept_desc>
  <concept_significance>100</concept_significance>
 </concept>
 <concept>
  <concept_id>10003033.10003083.10003095</concept_id>
  <concept_desc>Networks~Network reliability</concept_desc>
  <concept_significance>100</concept_significance>
 </concept>
</ccs2012>
\end{CCSXML}

\ccsdesc[500]{Software and its engineering~Software Bug Report Analysis}
%\ccsdesc[300]{Computer systems organization~Redundancy}
%\ccsdesc{Computer systems organization~Robotics}
%\ccsdesc[100]{Networks~Network reliability}

%
% Keywords. The author(s) should pick words that accurately describe the work being
% presented. Separate the keywords with commas.
\keywords{Bug Report, High-Impact Bug (HIB), Open-Source Projects, Prediction Models}

%
% A "teaser" image appears between the author and affiliation information and the body 
% of the document, and typically spans the page. 


%
% This command processes the author and affiliation and title information and builds
% the first part of the formatted document.
\maketitle

\section{Introduction}
Bug reports are the primary means through which developers triage and fix bugs~\cite{Bettenburg:20082}. To achieve this effectively, bug reports need to clearly describe those features that are important for the developers. However, previous studies have found that reporters do not always provide such features. Therefore, we first perform an exploratory study to identify the key features that reporters frequently miss in their initial bug report submissions. Then, we plan to propose an automatic approach for supporting reporters to make a good bug report. For our initial studies, we manually examine bug reports of five large-scale projects from two ecosystems such as Apache (Camel, Derby, and Wicket) and Mozilla (Firefox and Thunderbird). As initial results, we identify five key features that reporters often miss in their initial bug reports and developers require them for fixing bugs. We build and evaluate classification models using four different text-classification techniques. The evaluation results show that our models can effectively predict the key features. Our ongoing research focuses on developing an automatic features recommendation model to improve the contents of bug reports. 

\section{Background and Motivation}
Bug reports are the primary means through which developers triage and fix bugs. To achieve this effectively, bug reports need to clearly describe those features that are important for the developers. However, previous studies have found that reporters do not always provide such features. Therefore, we first perform an exploratory study to identify the key features that reporters frequently miss in their initial bug report submissions. Then, we plan to propose an automatic approach for supporting reporters to make a good bug report. For our initial studies, we manually examine bug reports of five large-scale projects from two ecosystems such as Apache (Camel, Derby, and Wicket) and Mozilla (Firefox and Thunderbird). As initial results, we identify five key features that reporters often miss in their initial bug reports and developers require them for fixing bugs. We build and evaluate classification models using four different text-classification techniques. The evaluation results show that our models can effectively predict the key features. Our ongoing research focuses on developing an automatic features recommendation model to improve the contents of bug reports. 
\section{Approach}
Bug reports are the primary means through which developers triage and fix bugs. To achieve this effectively, bug reports need to clearly describe those features that are important for the developers. However, previous studies have found that reporters do not always provide such features. Therefore, we first perform an exploratory study to identify the key features that reporters frequently miss in their initial bug report submissions. Then, we plan to propose an automatic approach for supporting reporters to make a good bug report. For our initial studies, we manually examine bug reports of five large-scale projects from two ecosystems such as Apache (Camel, Derby, and Wicket) and Mozilla (Firefox and Thunderbird). As initial results, we identify five key features that reporters often miss in their initial bug reports and developers require them for fixing bugs. We build and evaluate classification models using four different text-classification techniques. The evaluation results show that our models can effectively predict the key features. Our ongoing research focuses on developing an automatic features recommendation model to improve the contents of bug reports. 
\section{Evaluation and Future Work}
Bug reports are the primary means through which developers triage and fix bugs. To achieve this effectively, bug reports need to clearly describe those features that are important for the developers. However, previous studies have found that reporters do not always provide such features. Therefore, we first perform an exploratory study to identify the key features that reporters frequently miss in their initial bug report submissions. Then, we plan to propose an automatic approach for supporting reporters to make a good bug report. For our initial studies, we manually examine bug reports of five large-scale projects from two ecosystems such as Apache (Camel, Derby, and Wicket) and Mozilla (Firefox and Thunderbird). As initial results, we identify five key features that reporters often miss in their initial bug reports and developers require them for fixing bugs. We build and evaluate classification models using four different text-classification techniques. The evaluation results show that our models can effectively predict the key features. Our ongoing research focuses on developing an automatic features recommendation model to improve the contents of bug reports. 
\section{Conclusion}
Bug reports are the primary means through which developers triage and fix bugs. To achieve this effectively, bug reports need to clearly describe those features that are important for the developers. However, previous studies have found that reporters do not always provide such features. Therefore, we first perform an exploratory study to identify the key features that reporters frequently miss in their initial bug report submissions. Then, we plan to propose an automatic approach for supporting reporters to make a good bug report. For our initial studies, we manually examine bug reports of five large-scale projects from two ecosystems such as Apache (Camel, Derby, and Wicket) and Mozilla (Firefox and Thunderbird). As initial results, we identify five key features that reporters often miss in their initial bug reports and developers require them for fixing bugs. We build and evaluate classification models using four different text-classification techniques. The evaluation results show that our models can effectively predict the key features. Our ongoing research focuses on developing an automatic features recommendation model to improve the contents of bug reports. 


\section{Acknowledgments}

Identification of funding sources and other support, and thanks to individuals and groups that assisted in the research and the preparation of the work should be included in an acknowledgment section, which is placed just before the reference section in your document. 




%
% The next two lines define the bibliography style to be used, and the bibliography file.
\bibliographystyle{ACM-Reference-Format}
\bibliography{FSE}

% 
% If your work has an appendix, this is the place to put it.


\end{document}
